\documentclass[12pt]{report}
\usepackage{amsmath}
\usepackage{amssymb}
\usepackage{geometry}               
\usepackage{setspace}                
\usepackage{graphicx}
\usepackage{url}
\usepackage{hyperref}
%\usepackage{fullpage}

\newenvironment{myitemize}
{ \begin{itemize}
    \setlength{\itemsep}{5pt}
    \setlength{\parskip}{0pt}
    \setlength{\parsep}{0pt}     }
{ \end{itemize}                  } 


\begin{document}
\setlength\parindent{0pt}

\begin{center}
\Large
Syllabus for STAT 630: Statistical Methods\\
\large
Section 2, Fall 2021\\   
\end{center}
\vspace{5pt}

\normalsize
\textbf{Instructor:} Dr.~Eric Fox\\ 
\textbf{Office}: North Science 303A\\ 
\textbf{Email}: \url{eric.fox@csueastbay.edu}\\
%Phone: 510-885-4905 \\

\textbf{Lecture:} Tu/Th 8-9:40PM online using Zoom\\

\textbf{Office Hours (in-person):} Tu/Th 5-5:30 at N.~Science 303A, or by appointment\\

\textbf{Office Hours (online)}: M/W 3:30-5:30, or by appointment\\
Zoom link: \url{https://csueb.zoom.us/j/502694714}\\

\textbf{Website:} Course materials will be posted on Blackboard.\\ 

\textbf{Textbook}:  Diez, D., Barr, C. and Cetinkaya-Rundel M. \emph{OpenIntro Statistics}, 4th Edition, 2019. [Free PDF: \url{https://www.openintro.org/book/os/}]\\ 

\textbf{Additional Reference}: Chihara, L. and Hesterberg T. \emph{Mathematical Statistics with Resampling and R}. 2nd Edition, 2018.\\ 
Free electronic version: \url{http://library.csueastbay.edu/home}\\

\textbf{Software}:\\ 
R, can be downloaded here \url{https://www.r-project.org/}\\
RStudio, can be downloaded here \url{https://www.rstudio.com/}\\ 

\textbf{Course Topics}: This course will provide a graduate-level introduction to statistical methods and data science.  Topics include exploratory data analysis, statistical inference, and linear regression. Weekly computer labs will provide training in the use of the statistical programming language R.     
\begin{myitemize}
\item Data collection: sampling designs and experimental studies
\item Descriptive statistics and data visualization
\item Sampling distributions and the Central Limit Theorem
\item Confidence intervals 
\item Hypothesis testing
\item Resampling methods (the bootstrap, permutation tests)
\item Chi-square tests for goodness-of-fit and independence
\item Simple linear regression and correlation
\item Analyzing date-time and spatial data\\
%\item Unsupervised learning (heirarchical clustering and k-means)*
\end{myitemize} 

\textbf{Grading:}
There will be weekly homework assignments, and three take-home exams.  Both the homework and exams will be a combination of conceptual and data analysis problems.  The data analysis problems will require the use of R.  
\begin{myitemize}
\item 25\% Homework
\item 75\% Three Exams (25\% each)\\ 
\end{myitemize} 
  
\textbf{Policy on Late Assignments and Exams:}  Late homework will generally not be accepted.  However, your lowest scoring homework assignments will be dropped.  I may agree to extensions on due dates if you are experiencing an emergency or illness.\\

\textbf{Student Learning Outcomes:}  Upon successful completion of this course, students should be able to:
\begin{myitemize}
\item Apply statistical methodologies, including (a) summary statistics and graphical displays, (b) hypothesis testing and confidence intervals, and (c) linear regression and correlation.
\item Derive and understand basic theory underlying these methodologies. 
\item Use R and RStudio to analyze data sets and implement statistical methods.
\item Understand basic R programming, including vectors and data frames, subsetting, looping and control structures, simulation and resampling techniques. 
\item Communicate statistical concepts clearly and appropriately to others.\\ 
\end{myitemize}
\clearpage

\textbf{Technology Requirements:}  This course will use the web conferencing software Zoom.  To participate you will need a stable internet connection,  and a laptop or desktop computer equipped with a webcam, microphone, and speakers.  Please refer to the Zoom system requirements \href{https://support.zoom.us/hc/en-us/articles/201362023-System-requirements-for-Windows-macOS-and-Linux}{here}.\\

\textbf{Course Policies and Zoom Etiquette}:
\begin{myitemize}
\item All lectures will be delivered live during the scheduled class time, and attendance is highly recommended.  Recordings of the sessions will be posted on Blackboard for students that cannot attend or have connectivity issues.   
\item Make sure that your audio is muted upon entry into the class.
\item You may ask questions by using the chat function or by unmuting yourself.  Please try to not disrupt the instructor or other students.\\
\end{myitemize}

\textbf{Common Syllabus Items:}  Items such as policies on academic dishonesty, disability, and handling emergency situations can be found under ``University Policies" on Blackboard.\\

\textbf{A Note on Discrimination, Harassment, and Retaliation (DHR)}:\\
California State University East Bay is committed to a community free from sexual assault and violence. Title IX and CSU policy prohibit discrimination, harassment and retaliation, including Sex Discrimination, Sexual Harassment or Sexual Violence. CSUEB encourages anyone experiencing such behavior to report their concerns immediately. CSUEB has both confidential and non-confidential resources and reporting options available to you. \textbf{As a faculty member, I am required to report all incidents and thus cannot promise confidentiality.} I must provide our Title IX coordinator and or the DHR Administrator with relevant details such as the names of those involved in an incident. For confidential services, contact the \textbf{Confidential Advocate at 510-885-3700} or go to the Student Health and Counseling Center. For 24-hour crisis services call the Bay Area Women Against Rape (BAWAR) hotline at 510-845-7273. For more information about policies and resources or reporting options, please visit the following websites: \url{https://www.csueastbay.edu/diversity/title-ix/}


\end{document}