\documentclass[12pt, fleqn]{article}\usepackage[]{graphicx}\usepackage[]{color}
% maxwidth is the original width if it is less than linewidth
% otherwise use linewidth (to make sure the graphics do not exceed the margin)
\makeatletter
\def\maxwidth{ %
  \ifdim\Gin@nat@width>\linewidth
    \linewidth
  \else
    \Gin@nat@width
  \fi
}
\makeatother

\definecolor{fgcolor}{rgb}{0.345, 0.345, 0.345}
\newcommand{\hlnum}[1]{\textcolor[rgb]{0.686,0.059,0.569}{#1}}%
\newcommand{\hlstr}[1]{\textcolor[rgb]{0.192,0.494,0.8}{#1}}%
\newcommand{\hlcom}[1]{\textcolor[rgb]{0.678,0.584,0.686}{\textit{#1}}}%
\newcommand{\hlopt}[1]{\textcolor[rgb]{0,0,0}{#1}}%
\newcommand{\hlstd}[1]{\textcolor[rgb]{0.345,0.345,0.345}{#1}}%
\newcommand{\hlkwa}[1]{\textcolor[rgb]{0.161,0.373,0.58}{\textbf{#1}}}%
\newcommand{\hlkwb}[1]{\textcolor[rgb]{0.69,0.353,0.396}{#1}}%
\newcommand{\hlkwc}[1]{\textcolor[rgb]{0.333,0.667,0.333}{#1}}%
\newcommand{\hlkwd}[1]{\textcolor[rgb]{0.737,0.353,0.396}{\textbf{#1}}}%
\let\hlipl\hlkwb

\usepackage{framed}
\makeatletter
\newenvironment{kframe}{%
 \def\at@end@of@kframe{}%
 \ifinner\ifhmode%
  \def\at@end@of@kframe{\end{minipage}}%
  \begin{minipage}{\columnwidth}%
 \fi\fi%
 \def\FrameCommand##1{\hskip\@totalleftmargin \hskip-\fboxsep
 \colorbox{shadecolor}{##1}\hskip-\fboxsep
     % There is no \\@totalrightmargin, so:
     \hskip-\linewidth \hskip-\@totalleftmargin \hskip\columnwidth}%
 \MakeFramed {\advance\hsize-\width
   \@totalleftmargin\z@ \linewidth\hsize
   \@setminipage}}%
 {\par\unskip\endMakeFramed%
 \at@end@of@kframe}
\makeatother

\definecolor{shadecolor}{rgb}{.97, .97, .97}
\definecolor{messagecolor}{rgb}{0, 0, 0}
\definecolor{warningcolor}{rgb}{1, 0, 1}
\definecolor{errorcolor}{rgb}{1, 0, 0}
\newenvironment{knitrout}{}{} % an empty environment to be redefined in TeX

\usepackage{alltt}
\usepackage{amsmath}
\usepackage{amssymb}
\usepackage{geometry}
\usepackage{graphicx}
\usepackage{bm}
\usepackage{url}
\usepackage{hyperref}
\usepackage{enumerate}
\usepackage{fullpage}
\IfFileExists{upquote.sty}{\usepackage{upquote}}{}
\begin{document}

\setlength\parindent{0pt}

\begin{center}
\textbf{Lecture 10: Power and Sample Size Calculations}\\
\textbf{STAT 630, Fall 2021}\\
\hrulefill
\end{center}

\begin{table}[ht]
\begin{tabular}{l|l|l}
 & $H_0$ true & $H_A$ true\\
\hline
Reject $H_0$ & Type I error ($\alpha$) & Correct decision ($1-\beta$)\\
\hline
Do not reject $H_0$ & Correct decision & Type II error ($\beta$)\\
\end{tabular}
\end{table}

\begin{itemize}
\item Type I error: $\alpha = P( \text{Reject } H_0 | H_0 \text{ true})$
\begin{itemize}
\item the probability of falsely rejecting $H_0$
\end{itemize}
\item Type II error: $\beta = P( \text{Do not reject } H_0 | H_A \text{ true})$
\item Power: $1 - \beta = P(\text{Reject } H_0 | H_A \text{ true})$
\begin{itemize}
\item the probability of correctly rejecting $H_0$
\end{itemize}
\end{itemize}

Remarks:
\begin{itemize}
\item If we increase $\alpha$, then $\beta$ decreases (type 1 and 2 errors are inversely related).
\item If we increase the sample size $n$, then the power of the test increases (which implies the probability of a type 2 error $\beta$ decreases).\\
\end{itemize}
\vspace{10pt}

\textbf{Example}:  Blood pressure oscillates with the beating of the heart, and the systolic blood pressure is defined as the peak pressure when a person is at rest.  The average systolic blood pressure for people in the U.S. is about 130 mmHg with a standard deviation of about 25 mmHg.  We are interested in finding out if the average blood pressure of employees at a certain company is greater than the national average, so we randomly sample 100 employees and measure their systolic blood pressure.
\begin{enumerate}[(a)]
\item What are the null and alternative hypotheses?
\item Find the values of the sample mean $\bar{x}$ for which the null hypothesis would be rejected.  That is, find $c$ such that we reject $H_0$ if $\bar{x} > c$. Use $\alpha = 0.05$ significance level. 
\item Calculate the power of the test ($1-\beta$) if the true average blood pressure for employees at this company is 136 mmHg.
\item How large of a sample is needed to detect a 4 mmHg increase in average blood pressure with 0.9 power ($\beta=0.1$) and $\alpha = 0.05$?
\end{enumerate}
\clearpage

Here are the solutions using R:
\begin{knitrout}
\definecolor{shadecolor}{rgb}{0.969, 0.969, 0.969}\color{fgcolor}\begin{kframe}
\begin{alltt}
\hlcom{# part c}
\hlcom{# delta = 136-130 = 6}
\hlkwd{power.t.test}\hlstd{(}\hlkwc{n}\hlstd{=}\hlnum{100}\hlstd{,} \hlkwc{delta}\hlstd{=}\hlnum{6}\hlstd{,} \hlkwc{sd}\hlstd{=}\hlnum{25}\hlstd{,} \hlkwc{sig.level}\hlstd{=}\hlnum{0.05}\hlstd{,}
  \hlkwc{type}\hlstd{=}\hlstr{"one.sample"}\hlstd{,} \hlkwc{alternative}\hlstd{=}\hlstr{"one.sided"}\hlstd{)}
\end{alltt}
\begin{verbatim}
## 
##      One-sample t test power calculation 
## 
##               n = 100
##           delta = 6
##              sd = 25
##       sig.level = 0.05
##           power = 0.7699533
##     alternative = one.sided
\end{verbatim}
\begin{alltt}
\hlcom{# part d}
\hlkwd{power.t.test}\hlstd{(}\hlkwc{power}\hlstd{=}\hlnum{0.9}\hlstd{,} \hlkwc{delta}\hlstd{=}\hlnum{4}\hlstd{,} \hlkwc{sd}\hlstd{=}\hlnum{25}\hlstd{,} \hlkwc{sig.level}\hlstd{=}\hlnum{0.05}\hlstd{,}
  \hlkwc{type}\hlstd{=}\hlstr{"one.sample"}\hlstd{,} \hlkwc{alternative} \hlstd{=} \hlstr{"one.sided"}\hlstd{)}
\end{alltt}
\begin{verbatim}
## 
##      One-sample t test power calculation 
## 
##               n = 335.8827
##           delta = 4
##              sd = 25
##       sig.level = 0.05
##           power = 0.9
##     alternative = one.sided
\end{verbatim}
\end{kframe}
\end{knitrout}


\end{document}
